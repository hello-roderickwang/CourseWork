\begin{itemize} 
\item{The general system description: } 
Package delivery has always been a tricky task. Delivery men's job is to deliver packages as soon as possible. At the meantime, their employers expect them to use as few resources like gasline and vehicle abrasion as possible so that they can actually make more money. This is very similar to the traveling salesman problem (TSP). The TSP originally ask the following question: "Given a list of cities and the distances between each pair of cities, what is the shortest possible route that visits each city and returns to the origin city?" Isn't that similar with the package delivery problem as we just mentioned? The difference of these two problems is that each city in TSP doesn't actually affect later cost, but in package delivery problem, it does.
To understand the difference, we need to figure out what exactly is this package delivery problem we have been talking about. Assume a delivery man leaves from package warehouse and has 20 packages (P1 ~ P20) with him. He needs to go to 7 places (A ~ G) to deliver these packages. There are some roads (R1 ~ Rn) he can choose and those roads are not the same length. Considering the delivery man needs to deliver these packages as soon as possible, what is his best way to chose the route? Or even more, if we consider the extra package weight will cause vehicle consuming more gas, and delivery man needs to make sure he can deliver packages fast and economical as well. What route is best for this scenario?
\item{The real world scenarios: }
\begin{itemize} 
\item{Scenario1 description: }
Package delivery arrangement and optimization.
\item{System Data Input for Scenario1: }
Package weight (P1 ~ P20), Road length (R1 ~ Rn)
\item{Input Data Types for Scenario1: }
Matrix
\item{System Data Output for Scenario1: }
A functional road selection collection
\item{Output Data Types for Scenario1: }
Array
\item{Scenario2 description: }
Garbage collection arrangement and optimization.
\item{System Data Input for Scenario2: }
Garbage weight (P1 ~ P20), Road length (R1 ~ Rn)
\item{Input Data Types for Scenario2: }
Matrix
\item{System Data Output for Scenario2: }
A functional road selection collection
\item{Output Data Types for Scenario2: }
Array
\end{itemize}
\item{Project Time line and Divison of Labor.}
This project will be divided into 3 parts. First one is to try to build a working map to simulate the delivery progress. Secondly, we need to try to figure out what is the best strategy to deliver packages without considering gas consumption. Last part, we take package weight into consideration and try to analyze data to come up with a functional resolution. Each part will take about one week to finish. 
For now, we are thinking Xuenan and Zhenyuan will be in charge of coding and algorithm design, Kristie will be in charge of report writing and also algorithm improvements. Both three of us will be participating in presentation preparation.
\end{itemize}
